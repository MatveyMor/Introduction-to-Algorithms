\documentclass[12pt]{article}
\marginparwidth 0.5in 
\oddsidemargin 0.15in 
\evensidemargin 0.25in 
\marginparsep 0.25in
\topmargin 0in 
\textwidth 6in
\textheight 8 in
\headheight 0in
\headsep 0in
\parindent=15pt
\usepackage{amsmath}
\usepackage[english,russian]{babel}
\usepackage{mathtools}
\usepackage{amssymb}
\usepackage[T2A]{fontenc}
\usepackage[utf8]{inputenc}
\usepackage[russian]{babel}
\usepackage{graphicx}
\usepackage{hyperref}
\usepackage{listings}
\usepackage{tikz}

\begin{document}
	\author{Матвей Морозов, 678 группа}
	\title{Домашнее задание № 3.\\ Алгоритмы и модели вычислений}
	\maketitle
	
\section* {Задача 1}
Докажите, что следующие языки лежат в \textbf{NP}
\subsection* {Задача 1.1}
Язык описаний графов, у которых максимальная клика имеет размер не меньше k

{\itshape Решение:}
\\
\begin{enumerate}
	\item По определению \textbf{NP}
	\[L \in \textbf{NP} \Leftrightarrow L = \{x \in \Sigma^* | \exists y: |y| \le poly (|x|) \ \ \& \ \ R(x,y) = 1\}\]
	\item Пусть граф из $n = |V|$ задан матрицей смежности $n \times n$, то есть длина входа $n^2$. В качестве сертификата возьмём $k$ вершин $s_1, s_2, \dots, s_k$, которые составляют клику в исходном графе, то есть максимально полном подграфе.
	\item Проверять этот подграф на полноту будем так: $\forall i,j : i \not= j$ будем искать хотя бы один нуль в матрице. Если он будет в этой части матрицы смежности $\Rightarrow$ подграф не полный и размер клики явно меньше $k$, что означает, что слово не пренадлежит языку $L$ ну или же нам подсунули ложный сертификат. \\
	
	Если же в строках и столбцах матрицы смежности нет нулей, значит, что между любыми двумя вершинами существует ребро и это действительно клика размера $k$ $\Rightarrow$ слово принадлежит языку.
	
	\item Проверка $k \times k$ клеток матрицы смежности, где $k \le n$, осуществляется за $O(n^2)$, значит, языку принадлежит классу \textbf{NP}.
\end{enumerate}
\subsection* {Задача 1.2}
Задача проверки того, что два графа являются изоморфными.

{\itshape Решение:}
\\
Т.к. графы изоморфны, то существует биекция $V\Rightarrow V'$ и значит в качестве сертификата достаточно предъявить перестановку вершин, чтобы графы (и их матрицы смежности совпали). Графы A и B заданы их матрицами смежности a, b размера $n^2$, где n- количество вершин. Тогда для проверки мы просто за $O(poly(n^2))$ меняем местами строки и стобцы второй матрицы смежности в соответствии с перестановкой вершин и сравниваем ячейки в них. Если они полностью совпали - победа, после перестановки получился один и тот же граф, значит они были изоморфны.
\section*{Задача 2}
Покажите, что два определения класса \textbf{NP}, которые были даны на семинаре, эквивалентны.

{\itshape Решение:}
\\
\begin{enumerate}
	\item Запишем два определения класса \textbf{NP}
	\[\textbf{NP} = \bigcup_{k=1}^{\infty} NTime (n^k) \ \ (1)\]
	\[L \in \textbf{NP} \Leftrightarrow L = \{x \in \Sigma^* | \exists y: |y| \le poly (|x|) \ \ \& \ \ R(x,y) = 1\} \ \ (2)\]
	\item \textbf{(1)} $\Rightarrow$ \textbf{(2)}
	\\
	Пусть $L$ распознаётся недерминированной МТ $M$, которая работает за полиномиальное время, то есть за $O(n^c)$. В качестве сертификата $y$ возьмём последовательность значений функции перехода, а $R(x,y)$ пусть эмулирует $M$, используя данные $y$ для выбора одной из ветвей алгоритма.
	\item \textbf{(2)} $\Rightarrow$ \textbf{(1)}
	\\
	Построим недетерминированную МТ таким образом: сначала недетерминированно напишем на ленте наш сертификат $y$ : $|y| \le poly (|x|)$, затем запустим МТ $R(x,y)$.
	\item Честно, решение не мое, его подсмотрел {\href{http://ru.discrete-mathematics.org/fall2014/3/complexity/lecture-2-p-np.pdf}{{\itshape тут}}} (стр. 5)
\end{enumerate}
\newpage
\section* {Задача 5}
Постройте \textbf{NP}-сертификат простоты для числа $p = 3911, g = 13$. Простыми в рекурсивном построении считаются только числа $2,3,5$ (они сами являются своими сертификатами).

{\itshape Решение:}
\begin{enumerate}
	\item $p-1 = 3910 = 2 \cdot 5 \cdot 17 \cdot 23 = p_1^{k_1} \cdot p_2^{k_2} \cdot p_3^{k_3} \cdot p_4^{k_4}$\\
	
	Для $3911$ порождающий элемент по условию $13$.

	\item Числа $2$ и $5$ мы считаем простыми и на них наша рекурсия останавливается.
	\item $17-1 = 16 = 2^4$
	\\
	$23 - 1 = 22 = 2 \cdot 11$
	\item $11 - 1 = 10 = 2 \cdot 5$\\
	Сертификат:
\end{enumerate}
\begin{center}
	\begin{tikzpicture}[scale=0.2]
	\tikzstyle{every node}+=[inner sep=0pt]
	\draw [black] (39.7,-3.9) circle (3);
	\draw (39.7,-3.9) node {$3911$};
	\draw [black] (21.1,-19.5) circle (3);
	\draw (21.1,-19.5) node {$5$};
	\draw [black] (10.5,-18.9) circle (3);
	\draw (10.5,-18.9) node {$2$};
	\draw [black] (34.2,-19.5) circle (3);
	\draw (34.2,-19.5) node {$17$};
	\draw [black] (68.1,-12.3) circle (3);
	\draw (68.1,-12.3) node {$23$};
	\draw [black] (27.6,-32.5) circle (3);
	\draw (27.6,-32.5) node {$2$};
	\draw [black] (27.6,-32.5) circle (2.4);
	\draw [black] (36.5,-32.5) circle (3);
	\draw (36.5,-32.5) node {$2$};
	\draw [black] (36.5,-32.5) circle (2.4);
	\draw [black] (48.1,-33.3) circle (3);
	\draw (48.1,-33.3) node {$2$};
	\draw [black] (48.1,-33.3) circle (2.4);
	\draw [black] (17.5,-32.5) circle (3);
	\draw (17.5,-32.5) node {$2$};
	\draw [black] (17.5,-32.5) circle (2.4);
	\draw [black] (64.3,-22.5) circle (3);
	\draw (64.3,-22.5) node {$11$};
	\draw [black] (74.1,-21.4) circle (3);
	\draw (74.1,-21.4) node {$2$};
	\draw [black] (74.1,-21.4) circle (2.4);
	\draw [black] (59.3,-32.5) circle (3);
	\draw (59.3,-32.5) node {$2$};
	\draw [black] (59.3,-32.5) circle (2.4);
	\draw [black] (68.1,-32.5) circle (3);
	\draw (68.1,-32.5) node {$5$};
	\draw [black] (68.1,-32.5) circle (2.4);
	\draw [black] (37.03,-5.27) -- (13.17,-17.53);
	\fill [black] (13.17,-17.53) -- (14.11,-17.61) -- (13.65,-16.72);
	\draw [black] (37.4,-5.83) -- (23.4,-17.57);
	\fill [black] (23.4,-17.57) -- (24.33,-17.44) -- (23.69,-16.67);
	\draw [black] (38.7,-6.73) -- (35.2,-16.67);
	\fill [black] (35.2,-16.67) -- (35.94,-16.08) -- (34.99,-15.75);
	\draw [black] (42.58,-4.75) -- (65.22,-11.45);
	\fill [black] (65.22,-11.45) -- (64.6,-10.74) -- (64.31,-11.7);
	\draw [black] (31.83,-21.34) -- (19.87,-30.66);
	\fill [black] (19.87,-30.66) -- (20.81,-30.56) -- (20.19,-29.77);
	\draw [black] (32.84,-22.18) -- (28.96,-29.82);
	\fill [black] (28.96,-29.82) -- (29.77,-29.34) -- (28.87,-28.89);
	\draw [black] (34.72,-22.45) -- (35.98,-29.55);
	\fill [black] (35.98,-29.55) -- (36.33,-28.67) -- (35.35,-28.85);
	\draw [black] (36.33,-21.61) -- (45.97,-31.19);
	\fill [black] (45.97,-31.19) -- (45.76,-30.27) -- (45.05,-30.98);
	\draw [black] (69.75,-14.8) -- (72.45,-18.9);
	\fill [black] (72.45,-18.9) -- (72.43,-17.95) -- (71.59,-18.5);
	\draw [black] (67.05,-15.11) -- (65.35,-19.69);
	\fill [black] (65.35,-19.69) -- (66.1,-19.11) -- (65.16,-18.76);
	\draw [black] (62.96,-25.18) -- (60.64,-29.82);
	\fill [black] (60.64,-29.82) -- (61.45,-29.32) -- (60.55,-28.88);
	\draw [black] (65.37,-25.3) -- (67.03,-29.7);
	\fill [black] (67.03,-29.7) -- (67.22,-28.77) -- (66.28,-29.13);
	\end{tikzpicture}
\end{center}
\[13^{3910/2} \mod 3911 \not= 1\]
\[13^{3910/5} \mod 3911 \not= 1\]
\[13^{3910/17} \mod 3911 \not= 1\]
\[13^{3910/23} \mod 3911 \not= 1\]
Для $17$ выберем порождающий элемент $5$.
\[5^{16/2} \mod 17 \not= 1\]
Для $23$ выберем порождающий элемент $5$.
\[5^{22/11} \mod 23 \not= 1\]
\[5^{22/2} \mod 23 \not= 1\]
Для $11$ выберем порождающий элемент $2$.
\[2^{10/2} \mod 11 \not= 1\]
\[2^{10/5} \mod 11 \not= 1\]
Вычисления : \href{https://www.wolframalpha.com/input/?i=13%5E(3910%2F2)+mod+3911;+13%5E(3910%2F5)+mod+3911,+13%5E(3910%2F17)+mod+3911,+13%5E(3910%2F23)+mod+3911,++5%5E(16%2F2)+mod+17,++5%5E(22%2F11)+mod+23,+5%5E(22%2F2)+mod+23,+2%5E(10%2F2)+mod+11,+2%5E(10%2F5)+mod+11} {Вольфрам}
\section* {Задача 4}
Покажите, что классу \textbf{NP} принадлежит язык несовместных систем линейных уравнений с целыми коэффициентами от $2018$ неизвестных, и постройте соответствующий сертификат $y$ и проверяющий алгоритм $R(x,y)$.

{\itshape Решение:}
\\

По теореме Фредгольма $Ax=b - \text{ совместна} \Leftrightarrow  $ каждое решение сопряжённой однородной системы  $ c^TA = 0 \text{ удовлетворяло уравнению } c^Tb =0$. \\

Значит мы можем в качестве сертификата предъявлять такое решение $c_0$ сопряжённой однородной системы, что $c^T_0 b \ne 0$. Тогда проверка осуществляется подстановкой в расширенную матрицу системы $||A|B|| $ решение $c_0$. И далее операциями над матрицей убеждаемся в нарушении теоремы Фредгольма. \\

Находить $c_0$ можно решая методом Гаусса сопряжённую однородную систему. Причём это происходит за полином т.к. каждый элемент $c_0$ - полином от элементов исходной матрицы, степени не выше 2018 (из определения детерминанта). Размеры миноров ограничены 2018 (что есть константа). \\

Значит мы предъявили сертификат и алгоритм проверки, за полиномиальное время. Это и доказывает принадлежность языка \textbf{NP}.
\section* {Задача 3}
Покажите, что класс \textbf{NP} замкнут относительно $*$-операции Клини. Укажите, как построить для результирующего языка $L^*$, $L\in $ \text{NP} соответсвующий сертификат $y$ и проверяющий алгоритм $R(x,y)$.

{\itshape Решение:}
\\

В качестве сертификата $y$ достаточно взять разбиение слова из $L^*$ на подслова из $L$, и сертификаты слов из $L$, конкатенация которых образует исходное слово из $L^*$.
\\

В качестве алгоритма $R(x,y)$ мы будем запускать алгоритм $V(x,y)$ (который проверяет сертфикаты для $L$) на словах, которые образуют исходное слово из $L^*$, и их сертификатах.
\end{document}