\documentclass[11pt]{article}
\marginparwidth 0.5in 
\oddsidemargin 0.15in 
\evensidemargin 0.25in 
\marginparsep 0.25in
\topmargin 0in 
\textwidth 6in
\textheight 8 in
\headheight 0in
\headsep 0in
\parindent=15pt
\usepackage{amsmath}
\usepackage{mathtools}
\usepackage{amssymb}
\usepackage[T2A]{fontenc}
\usepackage[utf8]{inputenc}
\usepackage[russian]{babel}
\usepackage{graphicx}
\usepackage{hyperref}

\begin{document}
	\author{Матвей Морозов, 678 группа}
	\title{Домашнее задание № 2.\\ Алгоритмы и модели вычислений}
	\maketitle
	
	\textbf{Задача 2}
	\\
	$A$ -- мужик сказал, что выпала $6$.\\
	$B$ -- выпадает $6$.\\\\
	$\mathbb{P(A)} = \frac{1}{6}$, $\mathbb{P}(B) = \frac{1}{6}$, $\mathbb{P}(A|B) = \frac{3}{4}$
	\\\\
	По формуле Байеса: $\mathbb{P}(B|A) = \frac {\mathbb{P}(B)}{\mathbb{P(A)}}\cdot \mathbb{P}(A|B) = \frac{3}{4}$
	\\
	\\
	
	\textbf{Задача 3}
	\\
	$\mathbb{E}[\max\{X_1,X_2\}] + \mathbb{E}[\min\{X_1,X_2\}] = \mathbb{E}[\max\{X_1,X_2\} + \min\{X_1,X_2\}] = \mathbb{E}[X_1+X_2] =2 \cdot \mathbb{E}[X_1] = 7$
	\\\\\
	$\mathbb{E}[X_1] = \cfrac{1+2+3+4+5+6}{6} = \frac{7}{2}$
	\\
	\\
	
	\textbf{Задача 7}
	\\
	По определению события называются независимыми, если вероятность пересечения событий равна произведению вероятность каждого события.
	\\
	\\
	Вероятность выпадания чётного числа: $\mathbb{P}\{2,4,6\} = \frac{1}{2}$\\
	Вероятность выпадения кратного $3$ чётного числа: $\mathbb{P}\{6\} = \frac{1}{6}$\\
	Вероятность выпадания кратного $3$ числа: $\mathbb{P}\{3,6\} = \frac{1}{3}$\\
	\\
	$\mathbb{P}\{6\} = \mathbb{P}\{2,4,6\} \cdot \mathbb{P}\{3,6\}$\\
	\\
	Значит, события независимы.
	\newpage
	
	\textbf{Задача 1}
	\\
	$(i)$ Количество случаев, в которых число орлов $5$ равно $C_{10}^{5}$. Всего возможных комбинаций $2^{10}$.
	\\
	\\
	Вероятность выпадения равного числа орлов и решек $C_{10}^{5} \cdot 2^{-10}$
	\\\\
	$(ii)$ Вероятность того, что выпало не поровну $(1-C_{10}^{5} \cdot 2^{-10})$\\
	Вероятность того, что выпало больше орлов, чем решек, $\frac{1}{2}$\\\\
	Итого $\frac{1}{2}(1-C_{10}^{5} \cdot 2^{-10})$
	\\
	\\
	$(iii)$ Для первых $5$ бросков количество комбинаций $2^5$. Количество комбинаций для следующих $5$ бросков тоже $2^5$.
	\\\\
	Нас же интересует только $1$ из последних $2^5$ вариантов.\\\\
	Итого $2^{-5}$
\\\\

	\textbf{Задача 4}
	\\
	$1)$ $X$ -- количество бросков до выпадания двух $6$. \\
	$X_1$ -- количество бросков до выпадания одной $6$. \\
	$X_2$ -- количество бросков после $X_1$ до выпадания двух $6$.
	\\\\
	Пусть $X_1 = k$.
	\\\\
	\[\mathbb{P}[X_1] = \frac{1}{6} \cdot (\frac{5}{6})^{k-1} = \frac{5^{k-1}}{6^k}\]
	\[\mathbb{E}[X_1] = \sum_{k=1}^{\infty} k\frac{5^{k-1}}{6^k} = \sum_{k=1}^{\infty} (k+1)\frac{5^{k-1}}{6^k} - \sum_{k=1}^{\infty} \frac{5^{k-1}}{6^k} = \frac{6}{5} \sum_{k=1}^{\infty} k\frac{5^{k-1}}{6^k} - \frac{6}{5}\]
	\[\mathbb{E}[X_1] = \frac{6}{5}\mathbb{E}[X_1] - \frac{6}{5}\]
	\[\mathbb{E}[X_1] = 6\]
	По формуле полного математического ожидания
	$\mathbb{E}[X_2] = \frac{1}{6} \mathbb{E}[X_2|$после $X_1$ выпала $6] + \frac{5}{6}\mathbb{E}[X_2|$после $X_1$ не выпала $6] = \frac{1}{6} + \frac{5}{6}(1+\mathbb{E}[X]) = 1 + \frac{5}{6} \mathbb{E}[X]$
	\[\mathbb{E}[X] = 7 + \frac{5}{6} \mathbb{E}[X]\]
	\[\mathbb{E}[X] = 42\]
	\newpage
	$2)$
	
	\textbf{РОР:} \\\\
	$X_1$ -- количество бросков до выпадения первой решки. \\
	$X_2$ -- количество бросков после $X_1$ до выпадения орла. \\
	$X_3$ -- количество бросков после $X_2$ до выпадения решки.
	\[\mathbb{E}[X] = \mathbb{E}[X_1] + \mathbb{E}[X_2] + \mathbb{E}[X_3]\]
	\[\mathbb{E}[X_1] = \mathbb{E}[X_2] = 2\]
	По формуле полного математического ожидания
	$\mathbb{E}[X_3] = \frac{1}{2} \mathbb{E}[X_3|$после $X_2$ выпал орёл$] + \frac{1}{2}\mathbb{E}[X_3|$после $X_2$ выпала решка $] = \frac{1}{2} + \frac{1}{2}(1+\mathbb{E}[X]) = 1 + \frac{1}{2} \mathbb{E}[X]$
	\[\mathbb{E}[X] = 5 + \frac{1}{2} \mathbb{E}[X]\]
	\[\mathbb{E}[X] = 10\]
	\\
	\\
	
	\textbf{РРО:} \\\\
	$X_1$ -- количество бросков до выпадения первой решки. \\
	$X_2$ -- количество бросков после первой решки до выпадения второй решки. \\
	$X_3$ -- количество бросков после $X_2$ до выпадения орла.
	\[\mathbb{E}[X] = \mathbb{E}[X_1] + \mathbb{E}[X_2] + \mathbb{E}[X_3]\]
	\[\mathbb{E}[X_1] = 2\]
	\[\mathbb{E}[X_2] = 4\]
 	По формуле полного математического ожидания
 	$\mathbb{E}[X_3] = \frac{1}{2} \mathbb{E}[X_3|$после $X_2$ выпал орёл$] + \frac{1}{2}\mathbb{E}[X_3|$после $X_2$ выпала решка $] = \frac{1}{2} + \frac{1}{2}(1+\mathbb{E}[X_3]) = 1 + \frac{1}{2} \mathbb{E}[X_3]$
 	\[\mathbb{E}[X_3] = 2\]
 	\[\mathbb{E}[X] = 2+4+2 = 8\]
 	\\\\
 	Значит, раньше встретится РРО.
 	\newpage
 	
 	\textbf{Задача 6}
   	\\
   	Человек должен взять $n$ спичек из одной коробки и $n-k$ спичек из другой. Это можно сделать $C_{2n-k}^{k}$ вариантами. \\\\
   	Всего вариантов выбора $2n-k$ спичек $2^{2n-k}$.
   	\\
  	Итого: $\frac{C_{2n-k}^{k}}{2^{2n-k}}$
\end{document}